% !TEX root=../august-boeckh-hauptdokument.tex

\chapter[Glossar]{Glossar. Hier zeige ich, wie man eine Liste mit Namen erstellen kann, die man aus einem Text automatisch generiert.}
%---
%  Jetzt brauche ich die antiken Herrscher in vereinheitlichter Nennung, 
%  wobei bei der Erstnennung immer noch die Regierungsjahre dabei stehen sollen.
%  Am Ende bitte eine Liste mit Namen, Geburts- und Sterbedatum ausgeben lassen
%---

\gls{Balbinus}
Die Römische Republik befand sich in den letzten 100 Jahren ihrer Existenz, seit den Reformversuchen der Gracchen, in einer Phase des permanenten Bürgerkrieges. Octavian, der später \gls{Augustus} genannt wurde und sowohl Großneffe als auch Adoptivsohn Gaius Iulius Caesars war, hatte im Machtkampf im Anschluss an Caesars Ermordung zunächst dessen Mörder überwunden und anschließend seinen ehemaligen Kollegen im Triumvirat, Mark Anton, der angeblich gemeinsam mit Kleopatra von Ägypten aus ein hellenistisches Königreich zu errichten drohte, bei Actium 31 v. Chr. besiegt. \gls{Augustus} legte dann im Januar 27 v. Chr. seine im Bürgerkrieg errungene Alleinherrschaft vorgeblich nieder, doch ließ er sich dafür die Amtsvollmachten eines Volkstribunen und Oberbefehlshabers über die Legionen der Grenzprovinzen verleihen und periodisch erneuern, was künftig die formale Basis des Kaisertums war (siehe Prinzipat). Damit gelang ihm die Verrechtlichung seiner Macht.

%\printglossaries

\printglossary[
title={Liste der römischen Kaiser},
style = index]%alphabetische Ordnung

\section{Glossar aus csv-Datei generieren}

%--
% Geht das auch etwas automatisierter? 
% Kann man da nicht die Technik wie bei den Archäologen verwenden?
%--

Werde ich nie brauchen. 

Propagandistisch legitimierte er seinen Herrschaftsanspruch durch öffentliche und private Bauvorhaben, Schenkungen an die plebs, die Einbindung seiner Person in den beginnenden Kult und Verherrlichung des durch Beendigung der Bürgerkriege erreichten inneren Friedens in Architektur (Ara pacis) und Dichtung, die ihre klassische Blütezeit erfuhr. Den durch Bürgerkriege und Proskriptionen, später auch durch Umstrukturierungen kraft des Zensorenamtes personal stark veränderten Senat hatte Augustus durch Begünstigungen auf seine Seite gezogen: Die Nobilität wurde weitgehend entmachtet, behielt aber ihre herausgehobene soziale Position. Unter Ausschöpfung des verfassungsrechtlichen Spielraums hatte Augustus somit als erster Bürger Roms (princeps) die permanente Alleinherrschaft gewonnen und dabei den Fehler seiner Vorgänger vermieden, in den Verdacht zu geraten, die verhasste Königsherrschaft wiederherzustellen bzw. eine Tyrannis zu errichten. In seinem Tatenbericht (Res Gestae Divi Augusti) nennt Augustus sich an Ansehen (auctoritas) überlegen, an Amtsgewalt seinen Kollegen jedoch gleichgestellt. Dies war angesichts der Sondervollmachten und Machtmittel des princeps zwar eine Lüge, doch bildete diese Fiktion 300 Jahre lang die ideologische Basis der römischen Monarchie.

Die Stadt Rom wurde ihrer politischen Bedeutung entsprechend architektonisch und administrativ neu gestaltet, wie durch Herrschaftsanlagen, Tempelpflege, Spiele, Bäder sowie die Einrichtung einer Feuerwehrtruppe und einer mit polizeiähnlichen Aufgaben betrauten städtischen Garde, deren Oberbefehlshaber eine Art kaiserliche Stellvertreterposition einnahm. Auf sozialem Gebiet versuchte Augustus weitgehend erfolglos den Mitgliederrückgang der altadligen Patrizierfamilien durch verschärfte Ehegesetze zu lösen. Unter Augustus wurde das Reich auch durch formale Provinzialisierung von Ägypten und Eroberungen in der Alpenregion, Nordspanien sowie auf dem Balkan erweitert. Die Expansion in germanische Gebiete war bald nach der Niederlage des Varus im Jahre 9 abgeschlossen; die Gebiete zwischen Rhein und Elbe wurden nicht provinzialisiert, sondern von den Römern nur indirekt kontrolliert.

Seinen Stiefsohn und späteren Adoptivsohn \gls{Tiberius}, einen in die Ehe mitgebrachten Sohn seiner Frau Livia, schloss Augustus wohl zunächst von der Thronfolge aus (wenngleich das Kaisertum formal nie erblich war), da er ihm wichtige Ämter verweigerte. Augustus hätte einen blutsverwandten Nachfolger bevorzugt. Tiberius ging schließlich ins zeitweilige Exil nach Rhodos, um nicht beseitigt zu werden. Erst nach dem Tod von Augustus’ Neffen Marcellus, des zeitweilig zum Erben designierten Feldherrn Agrippa sowie der beiden Enkel Gaius und Lucius bestimmte Augustus Tiberius zum Nachfolger. Mögliche Zweifel an seiner Legitimation versuchte Tiberius durch demonstratives Zögern bei der Übernahme der mit dem Prinzipat verbundenen Ehren im Senat auszuräumen. Dennoch war das Verhältnis zwischen Kaiser und Senat gestört, so dass die senatorische Geschichtsschreibung Tiberius als Tyrannen schildert. Seine grausamen Charakterzüge sollen während seiner späten Regierungsjahre hervorgetreten sein, die durch angeblichen Hochverrat des Prätorianerpräfekten Lucius Aelius Seianus und die anschließenden Prozesse geprägt waren; die moderne Forschung hat dieses negative Bild in großen Teilen berichtigt.
%--
% Kann man das auch etwas automatisieren????
%--
Ein noch negativeres Bild zeichnen die Geschichtsschreiber vom dritten Kaiser \gls{Caligula}, auf dem nach Tiberius’ Tod große Hoffnungen ruhten, der aber, möglicherweise wegen seiner demonstrativen Hinwendung zum orientalischen Königtum, nach seiner Ermordung der Auslöschung des Andenkens verfiel und in der Historiographie als psychisch gestörter Sadist dargestellt wird. Die scheinbar pathologischen Handlungen Caligulas, der angeblich auch sein Lieblingspferd Incitatus in den Senatorenstand erheben wollte, werden in der modernen Forschung oft als Demütigungsrituale des nach Absolutismus strebenden Kaisers verstanden.

Claudius (41–54) war zunächst wegen sichtbarer körperlicher Behinderungen zugunsten Caligulas übergangen worden, war aber nach der Senatsrevolte, die zur Ermordung des Tyrannen führte, einziger legitimer Kandidat. Die Historiographie schildert ihn als introvertierten, seines hohen Amtes kaum fähigen Regenten, der sich geistigen Interessen hingab. In der modernen Forschung wird seine Regierung als eher erfolgreich bewertet, vor allem weil er die Grenzen stabilisierte und die Expansion zu einem Abschluss brachte. Kunstgeschichtliche Forschungen betonen die Einseitigkeit des überlieferten Bildes.

Ähnlich wie Caligula galt auch Nero (54–68), der durch seine ehrgeizige Mutter Agrippina intrigant zur Nachfolge geführt worden war, zunächst als Hoffnungsfigur. In den ersten fünf Regierungsjahren, die in der zeitgenössischen Literatur mit dem augusteischen Begriff des goldenen Zeitalters gewürdigt wurden, stand der jugendliche Nero unter dem Einfluss seines Erziehers, des Philosophen Seneca. Nero wird in der Historiographie als Tyrann und leidenschaftlicher Schauspieler dargestellt, der seine Mutter tötete. Nach der anschließenden Pisonischen Verschwörung mussten sich unter anderem Seneca, Lucan und Petronius das Leben nehmen. Nero wiederum wurde durch den Senat, der ihn zum Staatsfeind erklärt hatte, zum Selbstmord gezwungen. Er verfiel der senatorischen Verurteilung, so dass der Historiker Tacitus den Kaiser gerüchteweise als Urheber des großen Brandes in Rom nennt, den dieser zum Bau seiner Palastanlagen nutzte. Durch die anschließenden Christenverfolgungen, bei denen angeblich auch Paulus starb, ist seine Überlieferung in christlicher Zeit weiter in Misskredit geraten. Auch durch die althistorische Forschung wurde seine Regierungszeit eher negativ bewertet, was beispielsweise das Verhältnis zur senatorischen Oberschicht und die Vernachlässigung der Armee betraf.

Die Anfeindung Neros mit den beiden herrschaftslegitimierenden Gruppen, Senat und Heer, führte zur Delegitimation der julisch-claudischen Familie und in den Bürgerkrieg. Die bedeutende Rolle des Heeres zeigte sich im Vierkaiserjahr, in welchem sich die Generäle Galba, Otho und Vitellius als kurzzeitige Herrscher ablösten und aus dem schließlich Vespasian als Sieger hervorging. Nach seinem Familiennamen wird seine Dynastie die Flavier genannt.

Zusammenfassung:
Die Liste der römischen Kaiser der Antike enthält alle Kaiser des Römischen Reiches von Augustus, der im 1. Jahrhundert v. Chr. den Prinzipat begründete, bis Herakleios, dessen Herrschaftszeit 610–641 (ab 613 gemeinsam mit Konstantin III.) die späteste für das Ende der Antike in Betracht kommende Epochengrenze ist. Manche Forscher setzen frühere Endpunkte für die antike Kaiserzeit, etwa nach Theodosius I. (395), Romulus Augustulus (476), Justinian I. (565) oder Maurikios (602).

Die römische Kaiserzeit lässt sich grob in Prinzipat (einschließlich der Zeit der Reichskrise) und Spätantike einteilen (der Begriff Dominat gilt heute als veraltet). Die Liste überschneidet sich seit der Alleinherrschaft Konstantins I. 324–337 für etwa drei Jahrhunderte mit der Liste der byzantinischen Kaiser. Staatsrechtlich besteht kein Unterschied zwischen dem Römischen und dem Byzantinischen Reich, denn als byzantinisch bezeichnet erst die neuzeitliche Forschung den im Mittelalter zu einer griechisch geprägten Großmacht gewandelten Rumpf des römischen Weltreichs.

Das deutsche Wort Kaiser (wie auch das Wort Zar) leitet sich von Gaius Iulius Caesar ab, dem bekanntesten Träger des Cognomens Caesar.

%---
%  Hier die Liste mit den Kaisernamen drucken.
%---



