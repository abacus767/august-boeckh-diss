% !TEX root=../august-boeckh-hauptdokument.tex

\chapter[Infos]{Sich wiederholende Informationen mit leichter Änderungen erstelle ich gerne mit LuaTeX}
\section{Einstiegsbeispiel}


Bekannte Archäologen sind diese:
- Johann Joachim Winckelmann; geboren im Jahr 1717; gestorben 1768.
- Erich Boehringer; geboren im Jahr 1897; gestorben 1971.
% usw.

\DTLdisplaydb{archaeologen}%Zeige DB als Tabelle
\pagebreak
\DTLsort{Todesjahr=ascending}{archaeologen}%
\\
\DTLdisplaydb{archaeologen}

\begin{itemize}
\DTLforeach{archaeologen}
{\nachname=Nachname,%
\name=Name,%
\geburtsjahr=Geburtsjahr,%
\todesjahr=Todesjahr%
}
{\item \name\xspace \nachname;
geboren im Jahr \geburtsjahr\ifdefempty\todesjahr{.}{;
gestorben \todesjahr.}
}
\end{itemize}
%---
% Bitte hier eine Liste erstellen:
% Dann nach dem Todesjahr sotieren.
% Die Daten findest du in der csv-Datei archaeologen.csv
% ---



