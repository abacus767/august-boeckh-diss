% !TEX root=../august-boeckh-hauptdokument.tex

\chapter{Serienbriefe, die zweite}

\section{Studierende an der HU}

An der Humboldt-Universität zu Berlin gibt es viele Studierende:
Klassische Archäologie (54 im Bachelor, 23 im MA, 18 promovieren), 
Alte Geschichte (98 BA, 58 MA, 28 schreiben an der Doktorarbeit),
Klassische Philologie hat 78 Bachelors, 49 im Masterstudium und 23 stecken in der Promotion,
anders in der Ägyptologie: 49 im Grundstudium, 57 im Aufbaustudium und 14 herangehende Doktoren.

\section{Basic}


\DTLdisplaydb{fachstudenten}

\DTLforeach{fachstudenten}
{\fach=Fach,%
\ba=BA,%
\ma=MA,%
\phd=Phd%
}%
{\begin{description}
\item[\large\fach]~%Fake-Leerzeichen
\begin{labeling}{Promotion}
\item[Bachelor] \ba
\item[Master] \ma
\item[Promotion] \phd
\end{labeling}
\end{description}}

\section{Balkendiagram}

\begin{figure}
\DTLbarchart%
{variable=\ba,%Ausschlag auf y-Achse
barlabel=\fach,%Untere Beschriftung
upperbarlabel=\ba%Obere Beschriftung
}
{fachstudenten}%Datenbankname
{\ba=BA,%
\fach=Fach}
\caption{Anzahl der BA-Studierenden pro Fach}

\end{figure}

\begin{figure}[h]
	\DTLbarchart%
	{variable=\ma,%Ausschlag auf y-Achse
		barlabel=\fach,%Untere Beschriftung
		upperbarlabel=\ma%Obere Beschriftung
	}
	{fachstudenten}%Datenbankname
	{\ma=MA,%
		\fach=Fach}
	\caption{Anzahl der MA-Studierenden pro Fach}
	
\end{figure}

\begin{figure}[h]
	\DTLsetbarcolor{2}{Periwinkle}
	\DTLbarchart%
	{variable=\phd,%Ausschlag auf y-Achse
		barlabel=\fach,%Untere Beschriftung
		upperbarlabel=\phd%Obere Beschriftung
	}
	{fachstudenten}%Datenbankname
	{\phd=Phd,%
		\fach=Fach}
	\caption{Anzahl der PhD-Studierenden pro Fach}
	
\end{figure}
%----
% Diese Informationen möchte ich gerne auf unterschiedliche Arten zeigen:
% - zuerst als einfache Tabellen
% - Als Übersicht auf die einzelnen Fächer aufgeteilt
% - Als Tortendiagramm
% - Als Balkendiagramm wobei die Abschlüsse zusammengefasst werden
%---

\section{Tortendiagramm}

\begin{figure}[h]
\DTLsetpiesegmentcolor{3}{Periwinkle}
\DTLpiechart%
{variable=\ba,%
outerlabel=\fach,%
innerlabel={\DTLpiepercent\%},%
cutaway={2,4},%
rotateinner,%
rotateouter%
}
{fachstudenten}
{\ba=BA,%
\fach=Fach}
\caption{Tortendiagramm der BA'ler}
\end{figure}

\begin{figure}[h]
	\DTLsetpiesegmentcolor{3}{Periwinkle}
	\DTLsetpiesegmentcolor{1}{Plum}
	\DTLpiechart%
	{variable=\ma,%
		outerlabel=\fach,%
		innerlabel={\DTLpiepercent\%},%
		cutaway={1,3},%
		rotateinner,%
		}
	{fachstudenten}
	{\ma=MA,%
		\fach=Fach}
	\caption{Tortendiagramm der MA'ler}
\end{figure}


\section{Balkendiagramm für alle Studenten}

\begin{figure}[h]
\DTLsetbarcolor{3}{Salmon}
\DTLsetbarcolor{2}{Periwinkle}
\DTLsetbarcolor{1}{Plum}
\DTLbarwidth=.5cm
\DTLmultibarchart%
{variables={\ba,\ma,\phd},%
barlabel=\fach,%
verticalbars=false,%
uppermultibarlabels={BA (\ba), MA (\ma), PhD (\phd)}}
{fachstudenten}
{\ba=BA,%
\ma=MA,%
\phd=Phd,%
\fach=Fach}
\caption{Mehrere Balkendiagramme}
\end{figure}

\section{Archäologischer Katalog der Häuser von Pompeji}


%--
% In den Anhang/Appendix soll der Katalog.
% Alle Daten sind in der data/haeuser.csv-Datei
%---



