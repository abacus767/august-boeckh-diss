% !TEX root=../august-boeckh-hauptdokument.tex

\usepackage{babel}

\usepackage{graphicx}
\graphicspath{{figures/}}%relativer Pfad zu den Bildern
\usepackage{caption}
\usepackage{subcaption}%für Bilder nebeneinander

\usepackage{overpic}%Bilder "beschriften" bzw. etwas darauf legen.

\usepackage{tikz}%TikZ ist kein Zeichenprogramm.
\usetikzlibrary{spy}

\usepackage[
%content={Abbildung ausgeblendet, damit ich schneller kompilieren kann.},%Platzhaltertext
%filename,%Pfad anzeigen
size=normalsize,%Größe der Platzhalter
position=center,%Position des Platzhalters
noframe,%kein Rahmen
]{draftfigure}%modifizieren von ausgeblendeten Abbildungen

\usepackage{multicol}%mehrspaltiger Text

\widowpenalty=10000 %vermeide Schusterjungen
\clubpenalty=10000 %vermeide Hurenkinder

\usepackage{libertine}
\usepackage{xspace}
\usepackage{xcolor}

\usepackage{imakeidx}
\makeindex[
name=per,%Dies ist ein Personenindex
title = {Index antiker Personen},
columns=1,
options={-s style/index-style.tex},
]

\makeindex[
name=ant,
title = {Index antiker Texte},
columns=3,
options={-s style/index-style.tex},
]

\indexsetup{%
level=\addsec,%entspricht \section*
firstpagestyle=scrheadings, %normale Seitenformatierung
headers={\indexname}{\indexname}, %Kolumnentitel auf beiden Seiten
}

\usepackage[
series={},
nocritical,
noend,
noeledsec,
nofamiliar,
noledgroup
]{reledmac}%am besten für Übersetzungen
\usepackage{reledpar}

\usepackage{tabularx}%für Tabellen
\usepackage{booktabs}%für schöne Tabellen
\usepackage{tablefootnote}%für Fußnoten in Tabellen

\usepackage[headsepline=.5pt]
{scrlayer-scrpage}

\usepackage[singlelinecheck=false]{caption}

\usepackage{chngcntr}%Änderung von Nummerierungen
\counterwithout{table}{chapter}

\usepackage{marginnote}%für Marginalspalten

\usepackage{datatool}
\DTLloaddb%
{archaeologen}%Interner Name der Datenbank
{data/archaeologen.csv}%Speicherort

\DTLloaddb%lade Datenbank
{kaiser}%interner Name
{data/kaiser.csv}%Speicherort der Datei


\usepackage[section=subsection,%Glossar direkt nach Bezugstext
nonumberlist,%keine Seitenzahl
nopostdot,%kein Punkt ans Ende
]{glossaries}%für Glossare
\makeglossaries

\usepackage{SIunits}

\DTLloaddb{fachstudenten}
{data/fachstudenten.csv}

\DTLloaddb{haeuser}
{data/haeuser.csv}

\usepackage{databar}%für Balkendiagramme
\usepackage{datapie}%für Tortendiagramme

\usepackage{etoolbox}%für kondizionale Operatoren
\newtoggle{hund}%Definiert Bedingung.
\newtoggle{nocopyright}

\newcommand\missingcopyright{
\iftoggle{nocopyright}
{\setkeys{Gin}{draft}%setzt Bild auf draft
\setkeys{draftfigure}{%
filename = true,
content = {Abbildung musste aufgrund fehlender Digitalrechte ausgeblendet werden.}}}
{}}