% !TEX root=../august-boeckh-hauptdokument.tex

\iffalse%alles Nachfolgende wird ignoriert.
\newglossaryentry{Augustus}{name={\textbf{Augustus}},
	description={Augustus (Imperator Caesar Divi Filius Augustus),
		\newline *63\BC
		\newline gest. 14\AD
		\newline (27\BC -- 14\AD)},
	first={Augustus (27\protect\BC -- 14\protect\AD)}
}

\newglossaryentry{Mark Anton}{
	name={Mark Anton},
	description={Verräter},
	first={Mark Anton (der Verräter)}
}

\newglossaryentry{Kleopatra}{
	name={Kleopatra},%
	description={Kleopatra (Κλεοπάτρα Θεά Φιλοπάτωρ)
		\newline *69\BC
		\newline gest. 30\BC
		\newline aus dem Geschlecht der Ptolemäer, letzte der Pharaonen},%
	first={\textbf{Kleopatra}}
}
\fi%bis hierher wird ignoriert

\glssetexpandfield{name}
\glssetexpandfield{desc}
\glssetexpandfield{first}
\DTLforeach*{kaiser}%Datenbank aufrufen
{\kaiser=Kaiser,%
\name=Name,%
\geburt=Geburt,%
\geburtsort=Geburtsort,%
\tod=Tod,%
\sterbeort=Sterbeort,%
\reign=Regierungszeit,%
\shortreign=Regierungszeitkurz}%
{\newglossaryentry{\kaiser}%
{name={\textbf{\kaiser}},
description = {\kaiser
	\ifdefempty{\name}{}{\xspace(\name)};
\newline \geburt\xspace \ifdefempty{\geburtsort}{}{\xspace in \geburtsort}
\newline regierte\xspace\reign},
first = {\kaiser\ifdefempty{\shortreign}{}{\xspace\shortreign}}}}

