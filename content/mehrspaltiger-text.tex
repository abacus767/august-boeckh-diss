% !TEX root=../august-boeckh-hauptdokument.tex

\chapter{Mein mehrspaltiger Text}


%---
% Diesen Text bitte mehrspaltig setzen
%---
\begin{multicols}{2}[\section{Goethe – Werther}]

Eine wunderbare Heiterkeit hat meine ganze Seele eingenommen, gleich den süßen Frühlingsmorgen, die ich mit ganzem Herzen genieße. Ich bin allein und freue mich meines Lebens in dieser Gegend, die für solche Seelen geschaffen ist wie die meine. Ich bin so glücklich, mein Bester, so ganz in dem Gefühle von ruhigem Dasein versunken, daß meine Kunst darunter leidet. Ich könnte jetzt nicht zeichnen, nicht einen Strich, und bin nie ein größerer Maler gewesen als in diesen Augenblicken.\footnote{Komischer erster Absatz.}

Wenn das liebe Tal um mich dampft, und die hohe Sonne an der Oberfläche der undurchdringlichen Finsternis meines Waldes ruht, und nur einzelne Strahlen sich in das innere Heiligtum stehlen, ich dann im hohen Grase am fallenden Bache liege, und näher an der Erde tausend mannigfaltige Gräschen mir merkwürdig werden; wenn ich das Wimmeln der kleinen Welt zwischen Halmen, die unzähligen, unergründlichen Gestalten der Würmchen, der Mückchen näher an meinem Herzen fühle, und fühle die Gegenwart des Allmächtigen, der uns nach seinem Bilde schuf, das Wehen des Alliebenden,\footnote{Sehr schnulzig hier.} der uns in ewiger Wonne schwebend trägt und erhält; mein Freund!

Wenn's dann um meine Augen dämmert, und die Welt um mich her und der Himmel ganz in meiner Seele ruhn wie die Gestalt einer Geliebten - dann sehne ich mich oft und denke : ach könntest du das wieder ausdrücken, könntest du dem Papiere das einhauchen, was so voll, so warm in dir lebt, daß es würde der Spiegel deiner Seele, wie deine Seele ist der Spiegel des unendlichen Gottes! - mein Freund - aber ich gehe darüber zugrunde, ich erliege unter der Gewalt der Herrlichkeit dieser Erscheinungen.

Eine wunderbare Heiterkeit hat meine ganze Seele eingenommen, gleich den süßen Frühlingsmorgen, die ich mit ganzem Herzen genieße.\footnote{Ohje ohje, das ist ja mal ein Text, der es in sich hat und mich noch sehr beschäftigen wird, aber am Ende wird er mir vielleicht doch tatsächlich gefallen.} Ich bin allein und freue mich meines Lebens in dieser Gegend, die für solche Seelen geschaffen ist wie die meine. Ich bin so glücklich, mein Bester, so ganz in dem Gefühle von ruhigem Dasein versunken, daß meine Kunst darunter leidet. Ich könnte jetzt nicht zeichnen, nicht einen Strich, und bin nie ein größerer Maler gewesen als in diesen Augenblicken.

Wenn das liebe Tal um mich dampft, und die hohe Sonne an der Oberfläche der undurchdringlichen Finsternis meines Waldes ruht, und nur einzelne Strahlen sich in das innere Heiligtum stehlen, ich dann im hohen Grase am fallenden Bache liege, und näher an der Erde tausend mannigfaltige Gräschen mir merkwürdig werden; wenn ich das Wimmeln der kleinen Welt zwischen Halmen, die unzähligen, unergründlichen Gestalten der Würmchen…\footnote{Ein Würmchen, soso.}
\end{multicols}

 
%---
% Bitte griechisch links und deutsch rechts
%---
\section{Übersetzung „Griechisch / Deutsch“}

\begin{pairs}
\begin{Leftside}
\scriptsize
\beginnumbering
%\selectlanguage{greek}
\pstart
Μῆνιν ἄειδε θεὰ Πηληιάδεω Ἀχιλῆος\\ 
οὐλομένην, ἣ μυρί' Ἀχαιοῖς ἄλγε' ἔθηκε,\\ 
πολλὰς δ' ἰφθίμους ψυχὰς Ἄιδι προίαψεν\\ 
ἡρώων, αὐτοὺς δὲ ἑλώρια τεῦχε κύνεσσιν\\	
οἰωνοῖσί τε πᾶσι, Διὸς δ' ἐτελείετο βουλή,\\ 
ἐξ οὗ δὴ τὰ πρῶτα διαστήτην ἐρίσαντε\\ 
Ἀτρείδης τε ἄναξ ἀνδρῶν καὶ δῖος Ἀχιλλεύς.\\ 
\pend
\pstart
Τίς τάρ σφωε θεῶν ἔριδι ξυνέηκε μάχεσθαι;\\ 
Λητοῦς καὶ Διὸς υἱός· ὃ γὰρ βασιλῆι χολωθεὶς\\ 
νοῦσον ἀνὰ στρατὸν ὄρσε κακήν, ὀλέκοντο δὲ λαοί,\\ 
οὕνεκα τὸν Xρύσην ἠτίμασεν ἀρητῆρα \\
Ἀτρείδης·\\
\pend
\pstart
ὃ γὰρ ἦλθε θοὰς ἐπὶ νῆας Ἀχαιῶν\\
λυσόμενός τε θύγατρα φέρων τ' ἀπερείσι' ἄποινα,\\
στέμματ' ἔχων ἐν χερσὶν ἑκηβόλου Ἀπόλλωνος\\
χρυσέῳ ἀνὰ σκήπτρῳ, καὶ λίσσετο πάντας Ἀχαιούς,\\ 
Ἀτρείδα δὲ μάλιστα δύω, κοσμήτορε λαῶν·\\
Ἀτρείδαι τε καὶ ἄλλοι ἐυκνήμιδες Ἀχαιοί,\\
ὑμῖν μὲν θεοὶ δοῖεν Ὀλύμπια δώματ' ἔχοντες\\ 
ἐκπέρσαι Πριάμοιο πόλιν, εὖ δ' οἴκαδ' ἱκέσθαι·\\
\pend
\endnumbering
\end{Leftside}

\begin{Rightside}
\beginnumbering
\selectlanguage{ngerman}
\pstart
Singe den Zorn,\footnote{Also hier könnte man auch einfach ›Groll‹ sagen.} o Göttin, des Peleiaden Achilleus, 
Ihn, der entbrannt den Achaiern unnennbaren Jammer erregte,
Und viel tapfere Seelen der Heldensöhne zum Aïs
Sendete, aber sie selbst zum Raub darstellte den Hunden,
Und dem Gevögel umher. So ward Zeus Wille vollendet:
Seit dem Tag, als erst durch bitteren Zank sich entzweiten
Atreus Sohn, der Herrscher des Volks, und der edle Achilleus.
\pend
\pstart
Wer hat jene der Götter empört zu feindlichem Hader?
Letos Sohn und des Zeus. Denn der, dem Könige zürnend,
Sandte verderbliche Seuche durchs Heer; und es sanken die Völker:
Drum weil ihm den Chryses beleidigst, seinen Priester,
Atreus Sohn.
\pend
\pstart
Denn er kam zu den rüstigen Schiffen Achaias,
Frei zu kaufen die Tochter, und bracht' unendliche Lösung,
Tragend den Lorbeerschmuck des treffenden Phoibos Apollon
Und den goldenen Stab; laut flehte er zu den Achaiern,
Zu den Atreiden vor allen, den beiden Feldherrn der Völker: 
Atreus Söhn', und ihr andern, ihr hellumschienten Achaier,
Euch verleihn die Götter, olympischer Höhen Bewohner,
Priamos Stadt zu vertilgen, und wohl nach Hause zu kehren;
\pend
\endnumbering

\end{Rightside}
\end{pairs}

\Columns
%\Pages
%---
% Gibt es noch andere Möglichkeiten griechischen und deutschen Text nebeneinander zu setzen?? 
% Mit Zeilennummerierung?
%---



%---
% - Bitte nimm noch ein paar Formatierungen vor:
% - Fußnotengestaltung
% - Inhaltsverzeichnis schöner machen
% - Kolumnentitel setzen
% - Überschriften nicht als serifenlose Schrift
%---



